\documentclass[../main.tex]{subfiles}
\begin{document}
Field curvature, physical cuts, and stage coordinates being not always accurate are some of the problems that can alter the placement of features/objects in the sample and make stitching and consequently any automated analysis techniques very challenging. Coherent stitching of large samples is crucial for both visualization and automated analysis of samples. Conventional approaches that aim to stitch large volumetric data assumes linear transformations between adjacent tiles and solves a global/joint optimization for accurate placement of tiles [tera/fiji/citations inthem, \cite{tsai2011robust, chalfoun2017mist,bria2012terastitcher,emmenlauer2009xuvtools} ], but they do not take into account any non-linear deformations, such as physical sectioning with cuts or non-stationary objective curvature due to very long ( $> 1$ week) image acquisition. 



\subsection{Descriptor generation and match}
\subsection{Optimization}
\subsection{Render}
\subsection{Data share}
\subsection{Stitching-Results}
\begin{itemize}
\item show Johan's 2D projections for various samples
\end{itemize}

% \Blindtext
\end{document}
